\chapter{El Grupo de Lorentz}\label{appLor}
\section{Definici'on y generalidades}

En el contexto de la teor{'i}a Especial de la Relatividad (RE), un punto
(vector) de la variedad espacio-tiempo ({\em espacio de Minkowski}, $M_{4}$)
es caracterizado, en un sistema de referencia inercial (SRI) $O$, por las
coordenadas espacio-temporales cartesianas $x^{\mu
}=(x^0,x^1,x^2,x^3)=(t,x,y,z)=(t,x^{i})=(t,\vec{x})$.

Una Tranformaci'on de Lorentz (TL) es una transformaci'on lineal, real
y homog'enea de la forma 
\begin{equation}
x^{\mu }\rightarrow \widetilde{x}^{\mu }=\Lambda _{\nu }^{\mu }x^{\nu },
\label{tl}
\end{equation}
o en notaci'on matricial 
\begin{equation}
\widetilde{{\bf x}}={\bf \Lambda x,}
\end{equation}
tal que la forma cuadRatica 
\begin{equation}
s^2=\eta _{\mu \nu }x^{\mu }x^{\nu }=t^2-x^2-y^2-z^2,\qquad \eta
_{\mu \nu }={\rm diag}(1,-1,-1,-1),  \label{il}
\end{equation}
\newline
permanece invariante. Esto implica que las matrices ${\bf \Lambda }$ deben
satisfacer la condici'on 
\begin{equation}
\eta _{\mu \nu }\Lambda _{\lambda }^{\mu }\Lambda _{\rho }^{\nu }=\eta
_{\lambda \rho },  \label{indic}
\end{equation}
es decir, 
\begin{equation}
{\bf \Lambda }^{\top }{\bf \eta \Lambda =\eta }.  \label{condinv}
\end{equation}

F{\'{\i }}sicamente una TL de la forma (\ref{tl}) representa una
transformaci'on entre dos SRI's $O\rightarrow \widetilde{O}$ (vinculados
respectivamente a las coordenadas $x$ y $\widetilde{x}$). La condici'on
anterior expresa la equivalencia f{\'{\i }}sica de los sistemas de
referencia $O$ y $\widetilde{O}$ (ppio. de relatividad), y en particular la
existencia de una simetr{\'{\i }}a fundamental entre las tres dimensiones
espaciales y la dimensi'on temporal, las cual es manifestada en la
constancia de la velocidad (rapidez) de la luz en todos los SRI's\footnote{%
Aqu{\'{\i }} consideramos el sistema ``geometrizado'' de unidades, en el que 
$c\equiv 1$. ver \cite{MTW}, Cap{\'\i}tulo 1, box 1.8.}. El SRI $\widetilde{O%
}$ se mueve con {\em velocidad constante} con respecto a $O$. En general,
los ejes espaciales de $\widetilde{O}$ est'an rotados respecto de $O$.
Dado que una matriz $\Lambda $ determina totalmente una TL entre SRI's, es
com'un en la pr'actica identificar matrices y transformaciones.

El ejemplo m'as simple de TL corresponde a los llamados {\em boost},
donde los ejes espaciales de $\widetilde{O}$ son paralelos a los de $O$,
es decir no existe una rotaci'on espacial involucrada. La matriz ${\bf %
\Lambda }$ 
\begin{equation}
{\bf \Lambda }=\left( 
\begin{array}{cccc}
\gamma  & -\gamma v^1 & -\gamma v^2 & -\gamma v^3 \\ 
-\gamma v^1 & 1+\frac{(v^1)^2}{\left( v\right) ^2}(\gamma -1) & 
\frac{v^1v^2}{\left( v\right) ^2}(\gamma -1) & \frac{v^1v^3}{%
\left( v\right) ^2}(\gamma -1) \\ 
-\gamma v^2 & \frac{v^1v^2}{\left( v\right) ^2}(\gamma -1) & 1+\frac{%
(v^2)^2}{\left( v\right) ^2}(\gamma -1) & \frac{v^2v^3}{\left(
v\right) ^2}(\gamma -1) \\ 
-\gamma v^3 & \frac{v^1v^3}{\left( v\right) ^2}(\gamma -1) & \frac{%
v^2v^3}{\left( v\right) ^2}(\gamma -1) & 1+\frac{(v^3)^2}{\left(
v\right) ^2}(\gamma -1)
\end{array}
\right) ,  \label{mb}
\end{equation}
donde 
\begin{equation}
\gamma =\left[ 1-\left( v\right) ^2\right] ^{-1/2},\qquad v=\sqrt{%
\vec{v}\cdot \vec{v}}=\sqrt{%
(v^1)^2+(v^2)^2+(v^3)^2},\qquad \vec{v}=\left(
v^1,v^2,v^3\right) ,
\end{equation}
satisface la condici'on (\ref{condinv}) y conduce a la siguiente
transformaci'on de coordenadas inerciales 
\begin{equation}
\widetilde{\vec{x}}=\vec{x}+\frac{(\gamma -1)}{\left(
v\right) ^2}\left( \vec{v}\cdot \vec{x}\right) 
\vec{v}-\gamma \vec{v}t,\qquad \widetilde{t}=\gamma 
\left[ t-\left( \vec{v}\cdot \vec{x}\right) \right] .
\end{equation}

Esta TL determina las coordenadas espacio-temporales de un SRI $\widetilde{O}
$ que se mueve con rapidez (constante) $v$ respecto al SRI $O$, en la
direcci'on $\widehat{v}=\vec{v}/v$. Una transformaci'on del
tipo (\ref{mb}) es llamada un {\em boost} {\em en la direcci'on} $%
\widehat{v}$.

La condici'on (\ref{condinv}) implica que toda TL ${\bf \Lambda }$
satisface $\left( \det {\bf \Lambda }\right) ^2=1$, de modo que ${\bf %
\Lambda }$ necesariamente es no-singular y por lo tanto su matriz inversa $%
{\bf \Lambda }^{-1}$ siempre existe y tambi'en satisface (\ref{condinv}).
Si ${\bf \Lambda }^{\prime }$ es otra TL entonces el producto matricial $%
{\bf \Lambda \Lambda }^{\prime }$ tambi'en es una TL. En particular la
matriz identidad ${\bf 1}={\bf \Lambda }^{-1}{\bf \Lambda }={\bf \Lambda
\Lambda }^{-1}$ es la TL trivial. Estas propiedades, junto con la
asociatividad de la multiplicaci'on de matrices, muestran que las
matrices que satisfacen (\ref{condinv}) {\em forman un grupo} bajo la
multiplicaci'on matricial: el {\em Grupo de Lorentz }$L=O(1,3)$.

\subsection{Ejemplo: Transformaci'on del campo electromagn'etico}

Un cuadrivector $a^{\mu }$ es, {\em por definici'on}, un objeto que bajo
una TL, se transforma como 
\begin{equation}
a^{\mu }\rightarrow \widetilde{a}^{\mu }=\Lambda _{\nu }^{\mu }a^{\nu }. 
\end{equation}

An'alogamente, un tensor de segundo rango $a^{\mu \nu }$ se transforma 
\begin{equation}
a^{\mu \nu }\rightarrow \widetilde{a}^{\mu \nu }=\Lambda _{\lambda }^{\mu
}\Lambda _{\rho }^{\nu }a^{\lambda \rho }.  \label{tf}
\end{equation}

Este es el caso del tensor $F^{\mu \nu }$ que describe el campo electromagn%
'etico, representado matricialmente por 
\begin{equation}
{\bf F}=\left( 
\begin{array}{cccc}
0 & -E^1 & -E^2 & -E^3 \\ 
E^1 & 0 & -B^3 & B^2 \\ 
E^2 & B^3 & 0 & -B^1 \\ 
E^3 & -B^2 & B^1 & 0
\end{array}
\right) ,
\end{equation}
donde $\vec{E}$ es el vector campo el'ectrico y $%
\vec{B}$ el (pseudo-) vector campo magn'etico\footnote{%
Note que cuando aqu{'i} hablamos de $\vec{E}$ y $%
\vec{B}$ como ``(pseudo)vectores'' queremos decir que estos
objetos transforman (pseudo)vectorialmente {\em respecto a transformaciones
pertenecientes al (sub)grupo }$O(3)${\em \ de rotaciones}. $\vec{E%
}$ y $\vec{B}$ {\em no} son vectores respecto a TL.}

Bajo una TL ${\bf F}$ se transforma como 
\begin{equation}
\widetilde{{\bf F}}={\bf \Lambda F\Lambda }^{T}. 
\end{equation}

Usando (\ref{mb}) se encuentra que, bajo un boost, $\vec{E}$ y $%
\vec{B}$ se transforman como 
\begin{equation}
\widetilde{\vec{E}}=\gamma \vec{E}+\gamma \left( 
\vec{v}\cdot \vec{B}\right) -\frac{(\gamma -1)}{\left(
v\right) ^2}\left( \vec{v}\cdot \vec{E}\right) 
\vec{v}
\end{equation}
\begin{equation}
\widetilde{\vec{B}}=\gamma \vec{B}-\gamma \left( 
\vec{v}\cdot \vec{E}\right) -\frac{(\gamma -1)}{\left(
v\right) ^2}\left( \vec{v}\cdot \vec{B}\right) 
\vec{v}.
\end{equation}

\section{Descomposici'on del GL\label{descom}}

Si en (\ref{indic}) se hace $\lambda =\rho =0$, se obtiene que 
\begin{equation}
\left( \Lambda _{0}^0\right) ^2-\sum_{i=1}^3\left( \Lambda
_{0}^{i}\right) ^2=1, 
\end{equation}
de modo que 
\begin{equation}
\left( \Lambda _{0}^0\right) ^2=1+\sum_{i=1}^3\left( \Lambda
_{0}^{i}\right) ^2\geq 1.  \label{mayor}
\end{equation}

La ec. (\ref{mayor}) y el hecho que $\left( \det {\bf \Lambda }\right)
^2=1 $ implican que el {\em grupo completo de Lorentz} 
\begin{equation}
L=O(1,3)=\left\{ {\bf \Lambda }\in {\cal M}_{4\times 4}(R)\quad /\quad {\bf %
\Lambda }^{\top }{\bf \eta \Lambda =\eta }\right\} , 
\end{equation}
puede ser descompuesto en cuatro subconjuntos disconexos\footnote{%
Es decir, que no es posible conectar dos elementos cualesquiera de dos de
estos subconjuntos a trav'es de una curva continua en el espacio de los
elementos de $L$.}:

\begin{enumerate}
\item  Transformaciones de Lorentz {\em propias ortocronas} 
\begin{equation}
L_{+}^{\uparrow }=\left\{ {\bf \Lambda }\in O(1,3)\quad /\quad \Lambda
_{0}^0\geq 1,\quad \det {\bf \Lambda }=1\right\} .
\end{equation}

\item  Transformaciones de Lorentz {\em impropias ortocronas} 
\begin{equation}
L_{-}^{\uparrow }=\left\{ {\bf \Lambda }\in O(1,3)\quad /\quad \Lambda
_{0}^0\geq 1,\quad \det {\bf \Lambda }=-1\right\} .
\end{equation}

\item  Transformaciones de Lorentz {\em propias no ortocronas} 
\begin{equation}
L_{+}^{\downarrow }=\left\{ {\bf \Lambda }\in O(1,3)\quad /\quad \Lambda
_{0}^0\leq -1,\quad \det {\bf \Lambda }=1\right\} .
\end{equation}

\item  Transformaciones de Lorentz {\em impropias no ortocronas} 
\begin{equation}
L_{-}^{\downarrow }=\left\{ {\bf \Lambda }\in O(1,3)\quad /\quad \Lambda
_{0}^0\leq -1,\quad \det {\bf \Lambda }=-1\right\} .
\end{equation}
\end{enumerate}

Es 'util definir las TL's {\em propias}, $L_{+}=SO(1,3)=L_{+}^{\uparrow
}\cup L_{+}^{\downarrow }$; las transformaciones {\em impropias}, $%
L_{-}=L_{-}^{\uparrow }\cup L_{-}^{\downarrow }$; las transformaciones {\em %
ortocronas}, $L^{\uparrow }=L_{-}^{\uparrow }\cup L_{+}^{\uparrow }$; y las
transformaciones {\em no ortocronas}, $L^{\downarrow }=L_{-}^{\downarrow
}\cup L_{+}^{\downarrow }.$

$L_{-}$ y $L^{\downarrow }$ no constituyen grupos puesto que no contienen a
la identidad. Por el contrario, es f'acil verificar que $L_{+}$ y $%
L^{\uparrow }$ s{\'{\i }} forman grupos, que son subgrupos de $L=O(1,3)$.
Estos grupos reciben el nombre de {\em Grupo de Lorentz Propio} ($L_{+}$) y
{\em Grupo de Lorentz Ortocrono} ($L^{\uparrow }$). Finalmente $%
L_{+}^{\uparrow }$, la componente conexa del GL, define el {\em Grupo
Restringuido de Lorentz} (GL propio ortocrono), el cual es un {\em grupo de
Lie}\footnote{%
ver \cite{Gilmore}, Cap{\'\i}tulo 3.}.

La reflexi'on espacial (paridad) $x^{i}\rightarrow \widetilde{x}%
^{i}=-x^{i}$, $t\rightarrow \widetilde{t}=t$, que corresponde a una matriz $%
{\bf I}_{s}=diag(1,-1,-1,-1)$, es una TL impropia ortocrona, es decir ${\bf I}%
_{s}\in L_{-}^{\uparrow }$. An'alogamente la inversi'on temporal ${\bf %
I}_{t}=diag(-1,1,1,1)$ es una TL impropia no ortocrona, es decir ${\bf I}_{t}\in
L_{-}^{\downarrow }$. Es directo verificar que el elemento ${\bf I}_{st}=%
{\bf I}_{s}{\bf I}_{t}{\bf =I}_{t}{\bf I}_{s}\in L_{+}^{\downarrow }$.

Estos elementos, junto con la identidad, forman un subgrupo discreto
abeliano $R$ del GL. Este subgrupo $R$ es usualmente llamado {\em el grupo
de reflexiones espaciales y temporales}. En nuestro caso este grupo discreto
resulta ser el grupo factor $L/L_{+}^{\uparrow }$, obtenido a partir del
grupo de Lorentz $L$ y su parte conexa\footnote{Recu�rdese que en general,
el grupo factor de un grupo continuo $G$ por
su componente conexa $G_{0}$ es un {\em grupo discreto }${\mathit D=G/G}_{0}$.
ver \cite{Gilmore}, Cap�tulo 3.} $L_{+}^{\uparrow }$. Esto es una
manifestaci'on de la propiedad que permite descomponer un elemento
cualquiera de $L$ como la composici'on de un elemento perteneciente a $R$
y un elemento del grupo restringido $L_{+}^{\uparrow }$.

El subgrupo discreto $R$ queda definido por 
\begin{equation}
R=L/L_{+}^{\uparrow }=\left\{ {\bf 1},{\bf I}_{s},{\bf I}_{t},{\bf I}%
_{st}\right\} , 
\end{equation}
de modo que una transformaci'on ${\bf \Lambda }\in L$ puede siempre
escribirse como 
\begin{equation}
{\bf \Lambda }={\bf r}\cdot {\bf \Lambda }_{+}^{\uparrow }, 
\end{equation}
con apropiados elementos ${\bf r}\in R$ y ${\bf \Lambda }_{+}^{\uparrow }\in
L_{+}^{\uparrow }$.

Esto permite estudiar las propiedades del GL, analizando primero su parte
conexa $L_{+}^{\uparrow }$, a trav'es de su 'algebra $so(1,3)$, y
luego el grupo de reflexiones espaciales y temporales.

\section{Generadores y 'algebra $so(1,3)$\label{gene}}

La transformaci'on infinitesimal

\begin{equation}
{\bf \Lambda }={\bf 1+A},\qquad {\bf A}^2\approx {\bf 0}, 
\end{equation}
donde ${\bf A}$ satisface, de acuerdo con (\ref{condinv}), la condici'on 
\begin{equation}
({\bf \eta A})^{\top }+{\bf \eta A}={\bf 0,}  \label{anti1}
\end{equation}
implica que el 'algebra de Lie $so(1,3)$ es definida por 
\begin{equation}
so(1,3)=\left\{ {\bf A}\in {\cal M}_{4\times 4}(R)\quad /\quad ({\bf \eta A}%
)^{\top }+{\bf \eta A}={\bf 0}\right\} .  \label{alg}
\end{equation}

De (\ref{anti1}) se ve que la matriz ${\bf M\equiv \eta A}$ es
antisim'etrica, lo cual significa que una TL (propia ortocrona) queda
determinada por seis par'ametros independientes\footnote{${\bf A\equiv
\eta M}$, con ${\bf M}=$ matriz $4\times 4$ antisim'etrica y, por lo
tanto, con 6 elementos independientes.} de modo que el 'algebra del grupo
de Lorentz (AL) tiene dimensi'on 6, es decir, tiene 6 generadores. Por esto,
un elemento cualquiera de $so(1,3)$ es de la forma 
\begin{equation}
{\bf A=}\left( 
\begin{array}{cccc}
0 & \varepsilon ^{4} & \varepsilon ^{5} & \varepsilon ^{6} \\ 
\varepsilon ^{4} & 0 & -\varepsilon ^3 & \varepsilon ^2 \\ 
\varepsilon ^{5} & \varepsilon ^3 & 0 & -\varepsilon ^1 \\ 
\varepsilon ^{6} & -\varepsilon ^2 & \varepsilon ^1 & 0
\end{array}
\right) , 
\end{equation}
lo cual permite escribir una transformaci'on infinitesimal como\footnote{%
El factor $i$ se ha introducido por conveniencia.} 
\begin{equation}
\Lambda _{\nu }^{\mu }=\delta _{\nu }^{\mu }-i\varepsilon ^{\alpha }\left(
T_{\alpha }\right) _{\nu }^{\mu },\qquad \alpha =1,\ldots ,6 
\end{equation}
o matricialmente 
\begin{equation}
{\bf \Lambda =1}-i\varepsilon ^{\alpha }{\bf T}_{\alpha },\qquad \alpha
=1,\ldots ,6 
\end{equation}
donde $\varepsilon ^{\alpha }$ ($\alpha =1,\ldots ,6$) son los
par'ametros infinitesimales reales que determinan la TL y $\left(
T_{\alpha }\right) _{\nu }^{\mu }$ ('o ${\bf T}_{\alpha }$) son los
correspondientes generadores\footnote{%
Es usual en la literatura denotar los par'ametros por $\varepsilon
^{\lambda \rho }=-\varepsilon ^{\rho \lambda }$ y los generadores por $%
\left( T_{\lambda \rho }\right) _{\nu }^{\mu }=-\left( T_{\rho \lambda
}\right) _{\nu }^{\mu }$, de modo que la TL infinitesimal viene dada por $%
\Lambda _{\nu }^{\mu }=\delta _{\nu }^{\mu }-\frac{i}{2}\varepsilon
^{\lambda \rho }\left( T_{\lambda \rho }\right) _{\nu }^{\mu }$, donde $%
\left( T_{\lambda \rho }\right) _{\nu }^{\mu }=i\left[ \eta _{\rho \nu
}\delta _{\lambda }^{\mu }-\eta _{\lambda \nu }\delta _{\rho }^{\mu }\right] 
$. En esta notaci'on el 'algebra de Lie es de la forma $\left[ {\bf T}%
_{\mu \nu },{\bf T}_{\lambda \rho }\right] =-i\left( \eta _{\mu \lambda }%
{\bf T}_{\nu \rho }+\cdots \right) $ y los generadores de rotaciones y
boosts vienen dados por ${\bf J}^{i}=-\frac{1}{2}\epsilon ^{ijk}{\bf T}^{jk}$%
, ${\bf K}^{i}={\bf T}^{io}$. Sin embargo, esta notaci'on tiene el
inconveniente que induce al error de considerar la cantidad $\varepsilon
^{\lambda \rho }$ como un tensor de segundo rango antisim'etrico.
Recu'erdese que en este contexto la definici'on de tensor depende
justamente de las propiedades de transformaci'on de un objeto bajo TL's,
por lo que no es consistente aplicar esta definici'on a un objeto que,
precisamente, es usado para determinar las TL's. Por esto, aqu{\'{\i }}
introduciremos una notaci'on alternativa que es m'as ventajosa para
nuestros prop'ositos, pues permite visualizar m'as claramente la
estructura de las teor{\'{\i }}as que nos ocupan. Es 'util, sin embargo,
definir las ${\bf T}^{\mu \nu }$ como cantidades secundarias en algunos
c'alculos pr'acticos.}, los cuales
vienen dados por 
\begin{equation}
{\bf T}_{1}=\left( 
\begin{array}{cccc}
0 & 0 & 0 & 0 \\ 
0 & 0 & 0 & 0 \\ 
0 & 0 & 0 & -i \\ 
0 & 0 & i & 0
\end{array}
\right) ,\qquad {\bf T}_{1}=\left( 
\begin{array}{cccc}
0 & 0 & 0 & 0 \\ 
0 & 0 & 0 & i \\ 
0 & 0 & 0 & 0 \\ 
0 & -i & 0 & 0
\end{array}
\right) ,\qquad {\bf T}_{3}=\left( 
\begin{array}{cccc}
0 & 0 & 0 & 0 \\ 
0 & 0 & -i & 0 \\ 
0 & i & 0 & 0 \\ 
0 & 0 & 0 & 0
\end{array}
\right) ,  \label{d1}
\end{equation}
\begin{equation}
{\bf T}_{4}=\left( 
\begin{array}{cccc}
0 & i & 0 & 0 \\ 
i & 0 & 0 & 0 \\ 
0 & 0 & 0 & 0 \\ 
0 & 0 & 0 & 0
\end{array}
\right) ,\qquad {\bf T}_{5}=\left( 
\begin{array}{cccc}
0 & 0 & i & 0 \\ 
0 & 0 & 0 & 0 \\ 
i & 0 & 0 & 0 \\ 
0 & 0 & 0 & 0
\end{array}
\right) ,\qquad {\bf T}_{6}=\left( 
\begin{array}{cccc}
0 & 0 & 0 & i \\ 
0 & 0 & 0 & 0 \\ 
0 & 0 & 0 & 0 \\ 
i & 0 & 0 & 0
\end{array}
\right) ,  \label{d2}
\end{equation}
de modo que 
\begin{equation}
{\bf J}^{i}={\bf T}_{i},\qquad {\bf K}^{i}={\bf T}_{i+3},\qquad i=1,2,3, 
\end{equation}
donde $\vec{{\bf J}}$son los generadores de rotaciones y $%
\vec{{\bf K}}$ los de Boosts. As{\'{\i }}, las matrices ${\bf J}%
^{i}$ son herm{\'{\i }}ticas y las ${\bf K}^{i}$ antiherm{\'{\i }}ticas.

Estos resultados permiten verificar que los generadores del AL satisfacen
las relaciones de conmutaci'on 
\begin{eqnarray}
\left[ {\bf J}^{i},{\bf J}^{j}\right] &=&i\epsilon ^{ijk}{\bf J}^{k},
\label{allor1} \\
\left[ {\bf J}^{i},{\bf K}^{j}\right] &=&i\epsilon ^{ijk}{\bf K}^{k},
\label{allor2} \\
\left[ {\bf K}^{i},{\bf K}^{j}\right] &=&-i\epsilon ^{ijk}{\bf J}^{k},
\label{allor3}
\end{eqnarray}
y que una TL propia ortocrona arbitraria ${\bf \Lambda }_{+}^{\uparrow }$,
adopta la forma 
\begin{equation}
{\bf \Lambda }_{+}^{\uparrow }=\exp \left( -i\varepsilon ^{\alpha }{\bf T}%
_{\alpha }\right) ,  \label{exp}
\end{equation}
donde ahora los par'ametros $\varepsilon ^{\alpha }$ no son
necesariamente infinitesimales, es decir, la transformaci'on puede ser
finita. Por ejemplo, la transformaci'on (\ref{mb}) es obtenida haciendo $%
\varepsilon ^{i}=0$, $\varepsilon ^{i+3}=-\phi \widehat{v}^{i}$, con $\tanh
\phi=v$.

\section{Representaciones del GL\label{reppro}}

\subsection{Generalidades}

A conticuaci'on se presentar'an algunos elementos b'asicos
necesarios para construir las representaciones irreducibles {\em %
finito-dimensionales projectivas} del GL, las cuales ser'an de utililidad
en la descripci'on de las propiedades de transformaci'on de {\em campos%
} de materia. Una discusi'on acerca de las representaciones de
dimensi'on infinita (p.ej. unitarias), 'utiles en la descripci'on
de {\em estados de campos cu'anticos}, puede encontrarse en \cite
{BS,Cornwell}.

Una representaci'on de un grupo $G$ es una realizaci'on de la ley de
multiplicaci'on abstracta que define $G$, por medio de operadores, tales
como matrices u operadores diferenciales, que act'uan sobre alg'un
espacio vectorial de dimensi'on finita o infinita. En el caso del GL una 
{\em representaci'on finito-dimensional} puede expresarse a trav'es de
matrices que satisfacen la correspondiente regla de composici'on. En este
caso, a cada matriz ${\bf \Lambda }\in L$ le corresponde una matriz, que se
denotar'a por ${\bf S}({\bf \Lambda })$, tal que satisfacen la misma
regla de multiplicaci'on que las matrices ${\bf \Lambda }$. Si ${\bf %
\Lambda }_{1}$ y ${\bf \Lambda }_{2}$ son dos matrices que satisfacen (\ref
{condinv}), entonces las matrices ${\bf S(\Lambda )}$ constituyen una
representaci'on del GL si 
\begin{equation}
{\bf S}({\bf \Lambda }_{1}){\bf S}({\bf \Lambda }_{2})={\bf S}({\bf \Lambda }%
_{1}{\bf \Lambda }_{2}).  \label{repver}
\end{equation}

En Mec'anica Cu'antica Relativista (MCR), son adem'as de importancia, las {\em %
representaciones proyectivas}, es decir, representaciones salvo una fase. Las
matrices ${\bf S}$ constituyen una representaci'on proyectiva del GL si
satisfacen una regla de multiplicaci'on de la forma 
\begin{equation}
{\bf S}({\bf \Lambda }_{1}){\bf S}({\bf \Lambda }_{2})=e^{i\varphi ({\bf %
\Lambda }_{1},{\bf \Lambda }_{2})}{\bf S}({\bf \Lambda }_{1}{\bf \Lambda }%
_{2}),  \label{repproy}
\end{equation}
donde $\varphi ({\bf \Lambda }_{1},{\bf \Lambda }_{2})$ es alguna fase
(real) que puede depender en general de las transformaciones involucradas. As%
{\'{\i }} (\ref{repver}) es un caso particular de (\ref{repproy}).
Usualmente las matrices que satisfacen (\ref{repver}) son llamadas {\em %
representaciones univaluadas o verdaderas}, para diferenciarlas de las {\em %
representaciones proyectivas}$,$ que son entendidas como aquellas en que $%
\varphi \neq 0$.

\subsection{Representaciones Proyectivas de $L_{+}^{\uparrow }$}

Es un hecho conocido que no existe una relaci'on uno a uno entre grupos y
'algebras de Lie \cite{Gilmore}, debido a que grupos de Lie distintos
(p.ej. $SU(2)$ y $SO(3)$) pueden poseer la misma 'algebra\footnote{%
El 'algebra de Lie s'olo contiene informaci'on de las propiedades 
{\em locales} del Grupo.}. Sin embargo, al considerar representaciones
proyectivas de $L_{+}^{\uparrow }$ se simplifica en alguna medida el
trabajo, debido a que existe una relaci'on uno a uno\footnote{%
Por medio de exponenciaci'on, ver (\ref{exp}).} entre una 'algebra de
Lie y el correspondiente {\bf grupo covertor universal}\footnote{%
El grupo covertor universal de un grupo $G$ es el ('unico) grupo, que
tiene asociada la misma 'algebra que $G$, y es simplemente conexo \cite
{Gilmore}.} y, a su vez, entre las representaciones univaluadas del grupo
covertor y las representaciones proyectivas del grupo original.

En nuestro caso, el GL no es simplemente conexo\footnote{%
El subgrupo de rotaciones no es simplemente conexo, es doblemente conexo.
ver \cite{Ham}, cap'itulo 9.} y el grupo covertor universal de $%
L_{+}^{\uparrow }$ es $SL(2,C)$. Por lo tanto, al exponenciar {\bf %
representaciones del 'algebra de Lorentz} ($so(1,3)=sl(2,C)$) {\em no }se
obtienen en general representaciones univaluadas de $L_{+}^{\uparrow }$ sino
de $SL(2,C)$. Sin embargo, los objetos resultantes constituyen {\em %
representaciones proyectivas} de $L_{+}^{\uparrow }$, que son justamente las
que nos interesan. As\'{\i },\ el trabajo se reduce a construir
representaciones del {\em 'algebra} de Lorentz.

\subsection{El grupo $SL(2,C)$}

El grupo de Lie especial lineal bidimensional con par'ametros complejos, $%
SL(2,C)$, es definido como: 
\begin{equation}
SL(2,C)=\left\{ {\bf M}\in {\cal M}_{2\times 2}(C)\quad /\quad \det ({\bf M}%
)=1\right\} . 
\end{equation}

Una matriz ${\bf M}\in SL(2,C)$ puede ser escrita como 
\begin{eqnarray*}
{\bf M} &=&\left( 
\begin{array}{cc}
p_{0}+p_{3} & p_{1}+ip_{2} \\ 
p_{1}-ip_{2} & p_{0}-p_{3}
\end{array}
\right) \\
&=&p_{0}\left( 
\begin{array}{cc}
1 & 0 \\ 
0 & 1
\end{array}
\right) +p_{1}\left( 
\begin{array}{cc}
0 & 1 \\ 
1 & 0
\end{array}
\right) +p_{2}\left( 
\begin{array}{cc}
0 & i \\ 
-i & 0
\end{array}
\right) +p_{3}\left( 
\begin{array}{cc}
1 & 0 \\ 
0 & -1
\end{array}
\right) \\
&=&p_{0}{\bf 1}_{2}+p_{i}{\bf \sigma }^{i},
\end{eqnarray*}
donde ${\bf 1}_{2}$ es la matriz identidad bidimensional y ${\bf \sigma }^{i}
$ las matrices de Pauli. Los par'ametros $p$ son n'umeros complejos
que satisfacen la condici'on $\det ({\bf M}%
)=p_{0}^2-p_{1}^2-p_{2}^2-p_{3}^2=1$. Esto reduce a tres los
par'ametros complejos independientes que parametrizan una matriz de $%
SL(2,C)$.

Una matriz ${\bf M}\in SL(2,C)$ {\em que difiere infinitesimalmente de la
identidad} queda definida por tres par'ametros complejos infinitesimales
o, equivalentemente, por seis par'ametros infinitesimales reales. Esto
permite escribir 
\begin{equation}
{\bf M}={\bf 1}+p_{i}{\bf \sigma }^{i}={\bf 1}-i\varepsilon _{i}\left[ \frac{%
{\bf \sigma }^{i}}{2}\right] -i\varepsilon _{i+3}\left[ i\frac{{\bf \sigma }%
^{i}}{2}\right] ={\bf 1}-i\varepsilon ^{\alpha }{\bf L}_{\alpha },\qquad
\alpha =1,\ldots ,6, 
\end{equation}
donde ahora los par'ametros $p_{i}=\frac{1}{2}(\varepsilon
_{i+3}-i\varepsilon _{i})$ son infinitesimales y los $\varepsilon ^{\alpha }$
($\alpha =1,\ldots ,6$) son reales. Adem'as 
\begin{equation}
{\bf L}_{i}=\frac{{\bf \sigma }^{i}}{2}={\bf j}^{i},\qquad {\bf L}_{i+3}=i%
\frac{{\bf \sigma }^{i}}{2}={\bf k}^{i},  \label{gensl2c}
\end{equation}
son los generadores ${\bf L}_{\alpha }$ del 'algebra $sl(2,C)$, y
satisfacen la misma 'algebra que los operadores de rotaciones (${\bf J}%
^{i}$) y boosts (${\bf K}^{i}$) derivados en la secci'on \ref{gene}%
\footnote{%
y, como se veRa en la secci'on \ref{reps}, corresponden a la
representaci'on $(0,\frac{1}{2})$.}. Por lo tanto, los grupos de Lie $%
L_{+}^{\uparrow }$ y $SL(2,C)$ poseen la misma 'algebra de Lie, es decir $%
so(1,3)=sl(2,C)$. Sin embargo, el $L_{+}^{\uparrow }$ no es simplemente
conexo mientras que $SL(2,C)$ si lo es, y por lo tanto $SL(2,C)$ es el grupo
covertor universal de $L_{+}^{\uparrow }$. La construcci'on de elementos
finitos por medio de la exponenciaci'on de representaciones de elementos
del 'algebra de Lie $so(1,3)$ suministra representaciones del grupo $%
SL(2,C)$ que, en general, constituyen representaciones proyectivas de $%
L_{+}^{\uparrow }$.

Dado que $L_{+}^{\uparrow }$ y $SL(2,C)$ tienen la misma 'algebra de Lie,
ellos poseen la mismas propiedades {\em locales}\footnote{%
es decir, propiedades que involucran la composici'on de elementos cercanos a
la identidad.}. Sin embargo, las propiedades de globales de composici'on
no son las mismas. Estas caracter{\'{\i }}sticas pueden ejemplificarse
estudiando el comportamiento de ``rotaciones en torno del eje 3'', para TL's
y elementos de $SL(2,C)$.

Usando la matriz correspondiente al generador ${\bf J}^3$ en (\ref{d1}) es
directo verificar que una ``rotaci'on en un 'angulo de $2\pi $ en
torno de eje $3$'' es equivalente a la rotaci'on trivial, es decir 
\begin{equation}
{\bf \Lambda }=\exp (-i2\pi {\bf J}^3)={\bf 1}_{4}\in L_{+}^{\uparrow }. 
\end{equation}

Sin embargo, no ocurre lo mismo con elementos de $SL(2,C)$ ya que 
\begin{equation}
{\bf M}=\exp (-i2\pi {\bf j}^3)=\exp (-i\pi {\bf \sigma }^3)=-{\bf 1}%
_{2}\in SL(2,C). 
\end{equation}

As{\'{\i }}, es necesario que los par'ametros asociados a los generadores 
${\bf j}^{i}$ (``rotaciones'') var{\'{\i }}en en el intervalo $[0,4\pi ]$
para cubrir completamente los elementos de $SL(2,C)$. Mientras $%
L_{+}^{\uparrow }$ asocia a los 'angulos de rotaci'on $\theta $ y $%
\theta +2\pi $ el mismo elemento, el grupo $SL(2,C)$ les asocia elementos
distintos. Por esto, existe en general un \textit{mapeo dos a uno} (homomorfismo)
entre elementos de $SL(2,C)$ y elementos de $L_{+}^{\uparrow }$. A cada
elemento de $L_{+}^{\uparrow }$ le corresponden {\em dos} elementos de $%
SL(2,C)$: si ${\bf M}$ $\in SL(2,C)$ est'a asociado a ${\bf \Lambda }\in
L_{+}^{\uparrow }$ entonces $\left( -{\bf M}\right) $ $\in SL(2,C)$
tambi'en lo est'a. En el ejemplo anterior ${\bf M}={\bf 1}_{2}$ y $%
{\bf \Lambda }={\bf 1}_{4}$. Es f'acil ver que, debido a este
comportamiento, aunque las matrices en (\ref{gensl2c}) constituyen una {\em %
representaci'on del 'algebra} de $L_{+}^{\uparrow }$, las matrices
finitas construidas por exponenciaci'on de los generadores {\em no
constituyen una representaci'on de }$L_{+}^{\uparrow }$: en efecto, $%
e^{-i\pi {\bf j}^3}e^{-i\pi {\bf j}^3}=-{\bf 1}_{2}$, y una
representaci'on requiere una ley de multiplicaci'on de la forma $%
e^{-i\pi {\bf J}^3}e^{-i\pi {\bf J}^3}={\bf 1}$. Sin embargo, las
matrices as{\'{\i }} encontradas constituyen una {\em representaci'on
proyectiva de }$L_{+}^{\uparrow }$.

El grupo $L_{+}^{\uparrow }$ puede pensarse \cite{Gilmore} como el grupo
factor 
\begin{equation}
L_{+}^{\uparrow }=SL(2,C)/D, 
\end{equation}
donde $D$ es el {\em centro}\footnote{%
E.d., el conjunto de elementos que conmutan con todos los elementos del
grupo.} de $SL(2,C)$, que en nuestro caso est'a formado por los elementos
de $SL(2,C)$ asociados a la TL identidad, es decir, $D=\left\{ {\bf 1,-1}%
\right\} =Z_{2}$.

\subsection{Representaciones irreducibles finito-dimensionales
del 'algebra de Lorentz\label{rifdal}}

Una representaci'on matricial del AL $so(1,3)=sl(2,C)$, queda definida
por las matrices\footnote{%
Aqu{\'\i} denotamos los operadores ``abstractos'' (es decir no vinculados a
alguna representaci'on particular) por $J^{i}$ y $K^{i}$, mientras que
las correspondientes {\em matrices}, en una representaci'on
finito-dimensional dada, son denotadas por letras en negrita (p.ej. ${\bf J}%
^{i}$ y ${\bf K}^{i}$, 'o ${\bf j}^{i}$ y ${\bf k}^{i}$).} ${\bf J}^{i}$
y ${\bf K}^{i}$, $i=1,2,3$ que satisfacen las relaciones de conmutaci'on (%
\ref{allor1})-(\ref{allor3}). Si se introducen las combinaciones lineales
con coeficientes complejos 
\begin{equation}
{\bf X}^{i}\equiv \frac{1}{2}\left( {\bf J}^{i}+i{\bf K}^{i}\right) ,\qquad 
{\bf Y}^{i}\equiv \frac{1}{2}\left( {\bf J}^{i}-i{\bf K}^{i}\right) ,
\label{defxy}
\end{equation}
entonces (\ref{allor1})-(\ref{allor3}) implican que 
\begin{equation}
\left[ {\bf X}^{i},{\bf X}^{j}\right] =i\epsilon ^{ijk}{\bf X}^{k},\qquad 
\left[ {\bf Y}^{i},{\bf Y}^{j}\right] =i\epsilon ^{ijk}{\bf Y}^{k}, 
\end{equation}
\begin{equation}
\left[ {\bf X}^{i},{\bf Y}^{j}\right] =0. 
\end{equation}

Es decir, ${\bf X}^{i}$ e ${\bf Y}^{i}$ satisfacen el 'algebra $%
su(2)\oplus su(2)$. Esto permite construir representaciones irreducibles del
'algebra $so(1,3)$ a partir de productos directos de las bien conocidas
representaciones del 'algebra $su(2)$. Las representaciones irreducibles
construidas en esta forma quedan determinadas por los valores que asumen los
operadores de Casimir de cada 'algebra $su(2)$.

Consid'erese una representaci'on irreducible del 'algebra $su(2)$,
que seRa asociada a los operadores $X^{i}$. Las representaciones
irreducibles {\em finito-dimensionales} de $su(2)$ son herm{\'{\i }}ticas%
\footnote{$SU(2)$ es compacto.} y quedan determinadas por un n'umero $j$
que puede asumir valores {\em enteros o semienteros} $j=0,\frac{1}{2},1,%
\frac{3}{2},\ldots $. La representaci'on resultante, que se denotar'a
por $D_{X}^{(j)}$, es de dimensi'on $(2j+1)$ y el correspondiente espacio
de representaci'on es generado por la base $\left| j,m\right\rangle _{X}$%
, donde $m$ asume los valores $m=-j,-j+1,\ldots ,j-1,j$. Las matrices ${\bf x%
}^{i}$ quedan entonces definidas por: 
\begin{equation}
x_{(\pm )\widetilde{m}m}=\delta _{\widetilde{m},m\pm 1}\sqrt{\left( j\mp
m\right) \left( j\pm m+1\right) },\qquad x_{\widetilde{m}m}^3=m\delta _{%
\widetilde{m},m},  \label{x1}
\end{equation}
\begin{equation}
x_{\widetilde{m}m}^{i}=\left\langle j,\widetilde{m}\right| x^{i}\left|
j,m\right\rangle ,\qquad {\bf X}_{\pm }={\bf X}^1\pm i{\bf X}^2.
\label{x2}
\end{equation}

De la misma forma, es posible construir representaciones $D_{Y}^{(j^{\prime
})}$de dimensi'on $(2j^{\prime }+1)$ asociadas a los operadores $Y^{i}$,
de modo que la base del espacio de representaci'on viene dada por $\left|
j^{\prime },m^{\prime }\right\rangle _{Y}$, $m^{\prime }=-j^{\prime
},-j^{\prime }+1,\ldots ,j^{\prime }-1,j^{\prime }$. Consecuentemente, las
matrices ${\bf y}^{i}$ quedan definidas por 
\begin{equation}
y_{(\pm )\widetilde{m}^{\prime }m^{\prime }}=\delta _{\widetilde{m}^{\prime
},m^{\prime }\pm 1}\sqrt{\left( j^{\prime }\mp m^{\prime }\right) \left(
j^{\prime }\pm m^{\prime }+1\right) },\qquad y_{\widetilde{m}^{\prime
}m^{\prime }}^3=m^{\prime }\delta _{\widetilde{m}^{\prime },m^{\prime }},
\label{y1}
\end{equation}
\begin{equation}
y_{\widetilde{m}^{\prime }m}^{i}=\left\langle j^{\prime },\widetilde{m}%
^{\prime }\right| y^{i}\left| j^{\prime },m^{\prime }\right\rangle ,\qquad 
{\bf Y}_{\pm }={\bf Y}^1\pm i{\bf Y}^2.  \label{y2}
\end{equation}

Recu'erdese, por 'ultimo, que los operadores de Casimir de las
representaciones matriciales finito-dimensionales de $su(2)$ as{\'\i}
construidas son 
\begin{equation}
\left( {\bf x}\right) ^2={\bf x}^{i}{\bf x}^{i}=j(j+1){\bf 1},\qquad
\left( {\bf y}\right) ^2={\bf y}^{i}{\bf y}^{i}=j^{\prime }(j^{\prime }+1)%
{\bf 1}. 
\end{equation}

A partir de estos objetos es posible construir las matrices ${\bf X}^{i}$ e $%
{\bf Y}^{i}$ correspondientes a una representaci'on irreducible del
'algebra $so(1,3)$, denotada por $(j,j^{\prime })$. Estos operadores
deben satisfacer 'algebras $su(2)$ mutuamente conmutantes. Esto permite
construir las respectivas matrices de manera similar al usual procedimiento
de ``suma de momenta angulares''. Esto significa que el espacio
correspondiente a la representaci'on $(j,j^{\prime })$ de $so(1,3)$ tiene
como vectores base a 
\begin{equation}
\left| 
\begin{array}{c}
jj^{\prime } \\ 
mm^{\prime }
\end{array}
\right\rangle =\left| j,m\right\rangle _{X}\otimes \left| j^{\prime
},m^{\prime }\right\rangle _{Y}, 
\end{equation}
y las matrices ${\bf X}^{i}$ e ${\bf Y}^{i}$ vienen dadas por 
\begin{equation}
{\bf X}^{i}={\bf x}^{i}\otimes {\bf 1},\qquad {\bf Y}^{i}={\bf 1}\otimes 
{\bf y}^{i}.  \label{mayus}
\end{equation}

Por lo tanto, la representaci'on $(j,j^{\prime })$ es de dimensi'on $%
(2j+1)(2j^{\prime }+1)$. Finalmente, las matrices del 'algebra de Lorentz
buscadas, se calculan usando 
\begin{equation}
{\bf J}^{i}={\bf X}^{i}+{\bf Y}^{i},\qquad {\bf K}^{i}=i\left( {\bf Y}^{i}-%
{\bf X}^{i}\right) ,  \label{lk}
\end{equation}
lo que implica que en una representaci'on irreducible finito-dimensional $%
(j,j^{\prime })$ {\em las matrices de momentum angular son herm{\'\i}ticas},
mientras que {\em las matrices de boost son antiherm{\'\i}ticas}\footnote{%
El GL {\em no es compacto}. Esta propiedad es intuitivamente clara debido a
que la composici'on de boosts en una misma direcci'on y sentido es
equivalente a un boost con una velocidad resultante mayor (obtenida por la
expresi'on de adici'on relativista de velocidades). As{\'\i}, la
composici'on sucesiva de estos boosts converge a un ``boost con velocidad
de la luz''. Sin embargo, este objeto l{\'\i}mite, como puede verificarse
f'acilmente, {\em no} es una TL.}. Consecuentemente, las representaciones
del GL ser'an {\em no-unitarias}\footnote{%
Las matrices correspondientes a rotaciones puras son unitarias, no as{\'\i}
las que incluyen boosts.}.

Es 'util notar que, debido a que ${\bf L}^3{\bf =X}^3{\bf +Y}^3$,
el valor m'aximo que puede asumir (la tercera componente de) el momentum
angular en la representaci'on $(j,j^{\prime })$ es $j+j^{\prime } $.
Cuando se describen part{\'{\i }}culas usando campos de materia que se
transforman bajo una representaci'on de dimensi'on finita del GL, el
esp{\'{\i }}n de la part{\'{\i }}cula descrita viene dado justamente por el
momentum angular de la representaci'on correspondiente. Consecuentemente,
una representaci'on $(j,j^{\prime })$ describir'a part{\'{\i }}culas
de esp{\'{\i }}n $j+j^{\prime }$.

Es f'acil probar que, si $j+j^{\prime }$ es entero, la representaci'on
asociada resulta ser univaluada, mientras que si es semientero la
representaci'on ser'a proyectiva o {\em espinorial}\footnote{%
ver, por ejemplo, \cite{Cornwell}, p'agina 672.}.

De esta forma se han clasificado todas las representaciones proyectivas {\em %
finito-dimensionales irreducibles} del grupo restringuido de Lorentz ($%
L_{+}^{\uparrow }$). Si $\left( j_{1},j_{1}^{\prime }\right) $ y $\left(
j_{2},j_{2}^{\prime }\right) $ son dos representaciones irreducibles
entonces, la representaci'on $\left( j_{1},j_{1}^{\prime }\right) \otimes
\left( j_{2},j_{2}^{\prime }\right) $ puede descomponerse de la manera
siguiente\footnote{%
Aqu{\'\i} el s{\'\i}mbolo $\approx $ denota la {\em equivalencia} de
representaciones.} \cite{Cornwell} 
\begin{equation}
\left( j_{1},j_{1}^{\prime }\right) \otimes \left( j_{2},j_{2}^{\prime
}\right) \approx \sum_{j=\left| j_{1}-j_{2}\right|
}^{j_{1}+j_{2}}\sum_{j^{\prime }=\left| j_{1}^{\prime }-j_{2}^{\prime
}\right| }^{j_{1}^{\prime }+j_{2}^{\prime }}\left( j,j^{\prime }\right) . 
\end{equation}

Por ejemplo\footnote{%
Comparar con sec. \ref{vect}.}, 
\begin{equation}
\left( 0,\frac{1}{2}\right) \otimes \left( \frac{1}{2},0\right) \approx
\left( \frac{1}{2},\frac{1}{2}\right) , 
\end{equation}
\begin{equation}
\left( \frac{1}{2},\frac{1}{2}\right) \otimes \left( \frac{1}{2},\frac{1}{2}%
\right) \approx \left( 0,0\right) \oplus \left( 0,1\right) \oplus \left(
1,0\right) \oplus \left( 1,1\right) . 
\end{equation}

\subsection{Representaciones irreducibles: Casos particulares\label{reps}}

Con el fin de representar matricialmente los operadores, se considerar'an
los vectores que definen el espacio de representaci'on $\left(
j,j^{\prime }\right) $, ordenados como sigue : $\left| 
\begin{array}{c}
jj^{\prime } \\ 
j,j^{\prime }
\end{array}
\right\rangle $, $\ldots $, $\left| 
\begin{array}{c}
jj^{\prime } \\ 
-j,j^{\prime }
\end{array}
\right\rangle $, $\left| 
\begin{array}{c}
jj^{\prime } \\ 
j,j^{\prime }-1
\end{array}
\right\rangle $, $\ldots $, $\left| 
\begin{array}{c}
jj^{\prime } \\ 
-j,j^{\prime }-1
\end{array}
\right\rangle $, $\ldots $, $\left| 
\begin{array}{c}
jj^{\prime } \\ 
j,-j^{\prime }
\end{array}
\right\rangle $, $\ldots $, $\left| 
\begin{array}{c}
jj^{\prime } \\ 
-j,-j^{\prime }
\end{array}
\right\rangle $.

\subsubsection{Representaci'on $(0,0)$ (trivial)}

La representaci'on correspondiente a $j=j^{\prime }=0$ es la
representaci'on trivial. De las ecuaciones generales (\ref{x1})-(\ref{y2}%
) se observa que la representaci'on es unidimensional y que todos los
generadores son identicamente nulos. Por lo tanto, toda TL es representada
trivialmente por la identidad (unidimensional). Esta representaci'on es
de sp{\'\i}n cero y es usada para describir particular sin sp{\'\i}n
(escalares y speudo-escalares).

\subsubsection{Representaci'on $(\frac{1}{2},0)$ (izquierda)\label{left}}

Si $j={1}/{2}$, $j^{\prime }=0$ se obtiene una representaci'on de
spin $1/2$. Esta representaci'on es llamada ``izquierda'' debido
a que est'a ligada a la descripci'on de ``neutrinos de mano izquierda''.

En la representaci'on $({1}/{2},0)$ las matrices ${\bf x}^{i}$ son
bidimensionales, y las ${\bf y}^{i}$ son de dimensi'on uno e
id'enticamente nulas. La expresi'on expl{\'\i}cita de las matrices $%
{\bf x}^{i}$ quedan determinadas por (\ref{x1}) y (\ref{x2}). Usando la base 
$\left\{ \left| j,m\right\rangle _{X}\right\} $ dada por 
\begin{equation}
\left| \frac{1}{2},\frac{1}{2}\right\rangle _{X}=\left( 
\begin{array}{c}
1 \\ 
0
\end{array}
\right) ,\qquad \left| \frac{1}{2},-\frac{1}{2}\right\rangle _{X}=\left( 
\begin{array}{c}
0 \\ 
1
\end{array}
\right) , 
\end{equation}
se obtiene f'acilmente que 
\begin{equation}
{\bf x}^1=\frac{1}{2}\left( 
\begin{array}{cc}
0 & 1 \\ 
1 & 0
\end{array}
\right) ,\qquad {\bf x}^2=\frac{1}{2}\left( 
\begin{array}{cc}
0 & -i \\ 
i & 0
\end{array}
\right) ,\qquad {\bf x}^3=\frac{1}{2}\left( 
\begin{array}{cc}
1 & 0 \\ 
0 & -1
\end{array}
\right) 
\end{equation}
o en forma compacta 
\begin{equation}
{\bf x}^{i}=\frac{1}{2}{\bf \sigma }^{i} 
\end{equation}
donde ${\bf \sigma }^{i}$ son las usuales matrices de Pauli. Debido a que $%
{\bf y}^{i}=0$, (\ref{mayus}) implica ${\bf X}^{i}=\frac{1}{2}{\bf \sigma }%
^{i}$, ${\bf Y}^{i}=0$, de modo que (\ref{lk}) nos dice que los generadores
en la rep. $(\frac{1}{2},0)$ vienen dados por\footnote{%
Hemos incluido aqu{\'\i} el sub{\'\i}ndice $L$ para indicar que estas
matrices son la representaci'on de los generadores del GL en la
representaci'on izquierda.} 
\begin{equation}
({\bf J}_{L}{\bf )}^{i}=\frac{1}{2}{\bf \sigma }^{i},\qquad ({\bf K}_{L}{\bf %
)}^{i}=-\frac{i}{2}{\bf \sigma }^{i}. 
\end{equation}

\subsubsection{Representaci'on $(0,\frac{1}{2})$ (derecha)\label{right}}

An'alogamente al caso anterior, se obtiene que la rep. $(0,\frac{1}{2})$
es tambi'en bidimensional, pero ahora ${\bf X}^{i}=0$, ${\bf Y}^{i}=\frac{%
1}{2}{\bf \sigma }^{i}$. Por lo tanto, 
\begin{equation}
({\bf J}_{R}{\bf )}^{i}=\frac{1}{2}{\bf \sigma }^{i},\qquad ({\bf K}_{R}{\bf %
)}^{i}=\frac{i}{2}{\bf \sigma }^{i}. 
\end{equation}

\subsubsection{Representaci'on $(\frac{1}{2},\frac{1}{2})$ (vectorial)%
\label{vect}}

En el caso de la rep. $(\frac{1}{2},\frac{1}{2})$ las matrices resultantes
son de dimensi'on 4, al igual que la representaci'on ``original'' dada
por las matrices ($4\times 4$) $\Lambda _{\mu }^{\nu }$ que act'uan sobre
los vectores del espacio de Minkownski. M'as aun, 'estas son en
realidad la misma representaci'on\footnote{%
M'as exactamente, 'estas son representaciones equivalentes.}, debido a
que las matrices est'an ligadas por una transformaci'on de similitud.

Ya que $j=\frac{1}{2}$, las matrices ${\bf x}^{i}$ ser'an las mismas que
en el caso de la rep. $(\frac{1}{2},0)$, es decir, ${\bf x}^{i}=\frac{1}{2}{\bf %
\sigma }^{i}$. An'alogamente, $j^{\prime }=\frac{1}{2}$ implica que las
matrices ${\bf y}^{i}$ son las mismas que en $(0,\frac{1}{2})$, es decir,
tambi'en proporcionales a las matrices de Pauli: ${\bf y}^{i}=\frac{1}{2}%
{\bf \sigma }^{i}$. De este modo, (\ref{mayus}) nos dice que 
\begin{equation}
{\bf X}^{i}=\frac{1}{2}{\bf \sigma }^{i}\otimes {\bf 1}_{2},\qquad {\bf Y}%
^{i}=\frac{1}{2}{\bf 1}_{2}\otimes {\bf \sigma }^{i}, 
\end{equation}
y por lo tanto 
\begin{equation}
{\bf X}^1=\frac{1}{2}\left( 
\begin{array}{cccc}
0 & 1 & 0 & 0 \\ 
1 & 0 & 0 & 0 \\ 
0 & 0 & 0 & 1 \\ 
0 & 0 & 1 & 0
\end{array}
\right) ,\quad {\bf X}^2=\frac{1}{2}\left( 
\begin{array}{cccc}
0 & -i & 0 & 0 \\ 
i & 0 & 0 & 0 \\ 
0 & 0 & 0 & -i \\ 
0 & 0 & i & 0
\end{array}
\right) ,\quad {\bf X}^3=\frac{1}{2}\left( 
\begin{array}{cccc}
1 & 0 & 0 & 0 \\ 
0 & -1 & 0 & 0 \\ 
0 & 0 & 1 & 0 \\ 
0 & 0 & 0 & -1
\end{array}
\right) , 
\end{equation}
\begin{equation}
{\bf Y}^1=\frac{1}{2}\left( 
\begin{array}{cccc}
0 & 0 & 1 & 0 \\ 
0 & 0 & 0 & 1 \\ 
1 & 0 & 0 & 0 \\ 
0 & 1 & 0 & 0
\end{array}
\right) ,\quad {\bf Y}^2=\frac{1}{2}\left( 
\begin{array}{cccc}
0 & 0 & -i & 0 \\ 
0 & 0 & 0 & -i \\ 
i & 0 & 0 & 0 \\ 
0 & i & 0 & 0
\end{array}
\right) ,\quad {\bf Y}^3=\frac{1}{2}\left( 
\begin{array}{cccc}
1 & 0 & 0 & 0 \\ 
0 & 1 & 0 & 0 \\ 
0 & 0 & -1 & 0 \\ 
0 & 0 & 0 & -1
\end{array}
\right) . 
\end{equation}

Usando (\ref{lk}) se obtenen las matrices que representan a los generadores
de rotaciones y boosts en la representaci'on $(\frac{1}{2},\frac{1}{2})$: 
\begin{equation}
{\bf J}^1=\frac{1}{2}\left( 
\begin{array}{cccc}
0 & 1 & 1 & 0 \\ 
1 & 0 & 0 & 1 \\ 
1 & 0 & 0 & 1 \\ 
0 & 1 & 1 & 0
\end{array}
\right) ,\quad {\bf J}^2=\frac{1}{2}\left( 
\begin{array}{cccc}
0 & -i & -i & 0 \\ 
i & 0 & 0 & -i \\ 
i & 0 & 0 & -i \\ 
0 & i & i & 0
\end{array}
\right) ,\quad {\bf J}^3=\left( 
\begin{array}{cccc}
1 & 0 & 0 & 0 \\ 
0 & 0 & 0 & 0 \\ 
0 & 0 & 0 & 0 \\ 
0 & 0 & 0 & -1
\end{array}
\right) ,  \label{mm1}
\end{equation}
\begin{equation}
{\bf K}^1=\frac{1}{2}\left( 
\begin{array}{cccc}
0 & -i & i & 0 \\ 
-i & 0 & 0 & i \\ 
i & 0 & 0 & -i \\ 
0 & i & -i & 0
\end{array}
\right) ,\quad {\bf K}^2=\frac{1}{2}\left( 
\begin{array}{cccc}
0 & -1 & 1 & 0 \\ 
1 & 0 & 0 & 1 \\ 
-1 & 0 & 0 & -1 \\ 
0 & -1 & 1 & 0
\end{array}
\right) ,\quad {\bf K}^3=\left( 
\begin{array}{cccc}
0 & 0 & 0 & 0 \\ 
0 & i & 0 & 0 \\ 
0 & 0 & -i & 0 \\ 
0 & 0 & 0 & 0
\end{array}
\right) .  \label{mm2}
\end{equation}

Es f'acil verificar que los generadores en las expresiones (\ref{d1}) y (%
\ref{d2}) pueden obtenerse aplicando una transformaci'on de similitud
sobre las matrices aqu{\'{\i }} encontradas. Si se denotan las matrices en (%
\ref{mm1}) y (\ref{mm2}) por $\widetilde{{\bf T}}_{\alpha }$, entonces ${\bf %
T}_{\alpha }={\bf S}\widetilde{{\bf T}}_{\alpha }{\bf S}^{-1}$($\alpha
=1,\ldots ,6$), donde la matriz ${\bf S}$ puede elegirse como

\begin{equation}
{\bf S}=\left( 
\begin{array}{cccc}
0 & 1 & -1 & 0 \\ 
-1 & 0 & 0 & 1 \\ 
-i & 0 & 0 & -i \\ 
0 & 1 & 1 & 0
\end{array}
\right) . 
\end{equation}

En particular se verifica, por ejemplo, que 
\begin{eqnarray*}
{\bf T}_{3} &=&\allowbreak \left( 
\begin{array}{cccc}
0 & 0 & 0 & 0 \\ 
0 & 0 & -i & 0 \\ 
0 & i & 0 & 0 \\ 
0 & 0 & 0 & 0
\end{array}
\right) =\left( 
\begin{array}{cccc}
0 & 1 & -1 & 0 \\ 
-1 & 0 & 0 & 1 \\ 
-i & 0 & 0 & -i \\ 
0 & 1 & 1 & 0
\end{array}
\right) \left( 
\begin{array}{cccc}
1 & 0 & 0 & 0 \\ 
0 & 0 & 0 & 0 \\ 
0 & 0 & 0 & 0 \\ 
0 & 0 & 0 & -1
\end{array}
\right) \left( 
\begin{array}{cccc}
0 & 1 & -1 & 0 \\
-1 & 0 & 0 & 1 \\ 
-i & 0 & 0 & -i \\ 
0 & 1 & 1 & 0
\end{array}
\right) ^{-1} \\
&=&{\bf S}\widetilde{{\bf T}}_{3}{\bf S}^{-1}={\bf SJ}^3{\bf S}^{-1}.
\end{eqnarray*}

\subsection{Representaciones incluyendo reflexiones espaciales y temporales%
\label{riret}}

Consid'erese la extensi'on del grupo $SL(2,C)$, de modo que incorpore
reflexiones espaciales. Las representaciones del grupo ampliado nos entregan
representaciones (proyectivas) del grupo completo de Lorentz $L$. La
inclusi'on de estas transformaciones discretas no conduce a una 'unica
extensi'on de $SL(2,C)$, sino que a 8 extensiones inequivalentes\footnote{%
Esto complica en cierta medida el an'alisis.} \cite{Cornwell}. Aqu{\'\i}
s'olo se bosquejar'an algunos de los resultados principales, mostrando
los objetos que ser'an usados posteriormente.

Como es sabido\footnote{%
ver secci'on \ref{descom}.}, es posible descomponer una TL $\Lambda \in L$
como la composici'on de una reflexi'on espacial y/o temporal y un
elemento del grupo restringuido de Lorentz. Las propiedades algebr'aicas
de los elementos del grupo de reflexiones espaciales y temporales $R$ pueden
obtenerse directamente de las expresiones expl{\'{\i }}citas de ${\bf I}_{s}$%
, ${\bf I}_{t},$ ${\bf I}_{st}$, ${\bf J}_{i}$ y ${\bf K}_{i}$ (ver secs. 
\ref{descom} y \ref{gene}). Es directo verificar que 
\begin{equation}
{\bf I}_{s}^2={\bf I}_{t}^2={\bf I}_{st}^2={\bf 1},\qquad {\bf I}_{s}%
{\bf I}_{t}={\bf I}_{t}{\bf I}_{s}={\bf I}_{st}, 
\end{equation}
\begin{equation}
\left[ {\bf I}_{st}{\bf ,I}_{s}\right] =\left[ {\bf I}_{st},{\bf I}_{t}%
\right] ={\bf 0,\qquad }\left[ {\bf J}^{i}{\bf ,I}_{st}\right] =\left[ {\bf K%
}^{i}{\bf ,I}_{st}\right] ={\bf 0,} 
\end{equation}
\begin{equation}
\left[ {\bf J}^{i}{\bf ,I}_{s}\right] =\left[ {\bf J}^{i}{\bf ,I}_{t}\right]
={\bf 0,\qquad }\left\{ {\bf K}^{i}{\bf ,I}_{s}\right\} =\left\{ {\bf K}^{i}%
{\bf ,I}_{t}\right\} ={\bf 0,}  \label{conmutsp}
\end{equation}
de manera que 
\begin{equation}
{\bf X}^{i}{\bf I}_{s}={\bf I}_{s}{\bf Y}^{i},\qquad {\bf X}^{i}{\bf I}_{t}=%
{\bf I}_{t}{\bf Y}^{i},  \label{pasa1}
\end{equation}
\begin{equation}
{\bf Y}^{i}{\bf I}_{s}={\bf I}_{s}{\bf X}^{i},\qquad {\bf Y}^{i}{\bf I}_{t}=%
{\bf I}_{t}{\bf X}^{i},  \label{pasa2}
\end{equation}
\begin{equation}
\left[ {\bf X}^{i}{\bf ,I}_{st}\right] =\left[ {\bf Y}^{i}{\bf ,I}_{st}%
\right] ={\bf 0,}
\end{equation}
donde ${\bf X}^{i}$ e ${\bf Y}^{i}$ vienen dados por (\ref{defxy}).

Se asumir'a que el mapeo entre elementos de $SL(2,C)$ y $L_{+}^{\uparrow
} $ es tambi'en v'alido en el caso extendido. Denotando por $\left( 
{\bf P}_{+},{\bf P}_{-}\right) $, $\left( {\bf T}_{+},{\bf T}_{-}\right) $ e 
$\left( {\bf I}_{+},{\bf I}_{-}\right) $ a los elementos asociados a ${\bf I}%
_{s}$, ${\bf I}_{t}$, ${\bf I}_{st}$ en la extensi'on de $SL(2,C)$ se
tiene\footnote{%
En nuestro caso ${\bf P}_{-}=-{\bf P}_{+}$, ${\bf T}_{-}=-{\bf T}_{+}$ e $%
{\bf I}_{-}=-{\bf I}_{+}$.}, por ejemplo, que la relaci'on ${\bf I}%
_{s}^2={\bf 1}$ puede mapearse en ${\bf P}_{\pm }^2={\bf 1}_{2}$ o en $%
{\bf P}_{\pm }^2=-{\bf 1}_{2}$. Esta propiedad puede entenderse desde un
punto de vista algebraico notando que ${\bf I}_{s}^2={\bf 1}$ nos dice que 
${\bf I}_{s}^2\in SL(2,C)$ y que las relaciones (\ref{conmutsp}) implican $%
\left[ {\bf J}^{i}{\bf ,I}_{s}^2\right] =\left[ {\bf K}^{i}{\bf ,I}_{s}^2%
\right] ={\bf 0}$, por lo que ${\bf P}_{\pm }^2$ debe pertenecer al centro
de $SL(2,C)$, es decir ${\bf I}_{s}^2=\pm {\bf 1}_{2}$. Lo mismo es v'alido
para ${\bf I}_{t}^2$ y ${\bf I}_{st}^2$.

Por lo tanto, si se incluyen las transformaciones discretas de Lorentz,
existen $2^3=8$ extensiones distintas de $SL(2,C)$: 
\begin{equation}
{\bf P}_{\pm }^2=\epsilon _{s}{\bf 1}, \qquad {\bf T}_{\pm
}^2=\epsilon _{t}{\bf 1}, \qquad {\bf I}_{\pm }^2=\epsilon _{st}%
{\bf 1},  \label{cua}
\end{equation}
donde $\epsilon _{s}$, $\epsilon _{t}$ y $\epsilon _{st}$, pueden asumir los
valores $\pm 1$.

Es posible asumir, sin perder generalidad\footnote{%
ver \cite{Cornwell}, p'agina 677.}, que 
\begin{equation}
{\bf P}_{+}{\bf T}_{+}={\bf I}_{+}, 
\end{equation}
y entonces (\ref{cua}) implica \cite{BS} que 
\begin{equation}
{\bf T}_{+}{\bf P}_{+}=\epsilon _{s}\epsilon _{t}\epsilon _{st}{\bf I}_{+}.
\label{reves}
\end{equation}

As{\'{\i }}, por ejemplo, si $\epsilon _{s}=\epsilon _{s}=\epsilon _{s}=1$
entonces ${\bf T}_{+}{\bf P}_{+}={\bf I}_{+}$, mientras que si $\epsilon
_{s}=\epsilon _{s}=-\epsilon _{s}=1$ entonces ${\bf T}_{+}{\bf P}_{+}=-{\bf I%
}_{+}={\bf I}_{-}$. En la tabla siguiente se muestran las posibles
extensiones $\widetilde{L}_{i}$ ($i=1,\ldots ,8$) de $SL(2,C)$.

\begin{center}
\begin{tabular}{c||c|c|c|c}
extensi'on & $\epsilon _{s}$ & $\epsilon _{t}$ & $\epsilon _{st}$ & ${\bf %
T}_{+}{\bf P}_{+}$ \\ \hline\hline
$\widetilde{L}_{1}$ & $1$ & $1$ & $1$ & ${\bf I}_{+}$ \\ \hline
$\widetilde{L}_{2}$ & $1$ & $1$ & $-1$ & ${\bf I}_{-}$ \\ \hline
$\widetilde{L}_{3}$ & $1$ & $-1$ & $1$ & ${\bf I}_{-}$ \\ \hline
$\widetilde{L}_{4}$ & $1$ & $-1$ & $-1$ & ${\bf I}_{+}$ \\ \hline
$\widetilde{L}_{5}$ & $-1$ & $1$ & $1$ & ${\bf I}_{-}$ \\ \hline
$\widetilde{L}_{6}$ & $-1$ & $1$ & $-1$ & ${\bf I}_{+}$ \\ \hline
$\widetilde{L}_{7}$ & $-1$ & $-1$ & $1$ & ${\bf I}_{+}$ \\ \hline
$\widetilde{L}_{8}$ & $-1$ & $-1$ & $-1$ & ${\bf I}_{-}$ \\ \hline
\end{tabular}
\end{center}

De estas 8 extensiones posibles, la m'as usada en la descripci'on
relativista de part{\'{\i }}culas es $\widetilde{L}_{8}$ \cite{Cornwell}%
\footnote{%
En el caso de part{\'\i}culas de sp{\'\i}n $1/2$, $\widetilde{L}_{8}$
describe electrones y neutrinos, a trav'es de las ecuaciones de Dirac y
Weyl.}. En lo que sigue {\em s'olo se considerar'a esta extensi'on}.

Si una representaci'on de $\widetilde{L}_{8}$ contiene, en su
reducci'on a los elementos de $SL(2,C)$, la representaci'on $\left(
j,j^{\prime }\right) $, entonces necesariamente debe contener adem'as la
representaci'on $\left( j^{\prime },j\right) $ ya que, por ejemplo, las
primeras relaciones en (\ref{pasa1}) y (\ref{pasa2}) implican que 
\begin{equation}
{\bf X}^3\left( {\bf I}_{s}\left| 
\begin{array}{c}
jj^{\prime } \\ 
mm^{\prime }
\end{array}
\right\rangle \right) =m^{\prime }\left( {\bf I}_{s}\left| 
\begin{array}{c}
jj^{\prime } \\ 
mm^{\prime }
\end{array}
\right\rangle \right) ,\qquad m^{\prime }=-j^{\prime },\ldots ,j^{\prime }, 
\end{equation}
\begin{equation}
{\bf Y}^3\left( {\bf I}_{s}\left| 
\begin{array}{c}
jj^{\prime } \\ 
mm^{\prime }
\end{array}
\right\rangle \right) =m\left( {\bf I}_{s}\left| 
\begin{array}{c}
jj^{\prime } \\ 
mm^{\prime }
\end{array}
\right\rangle \right) ,\qquad m=-j,\ldots ,j, 
\end{equation}
de modo que el vector $\left( {\bf I}_{s}\left| 
\begin{array}{c}
jj^{\prime } \\ 
mm^{\prime }
\end{array}
\right\rangle \right) $ debe ser proporcional a $\left| 
\begin{array}{c}
j^{\prime }j \\ 
m^{\prime }m
\end{array}
\right\rangle $. En otras palabras, la transformaci'on espacial ${\bf I}%
_{s}$ mapea vectores del espacio de representaci'on de $\left(
j,j^{\prime }\right) $ en los de $\left( j^{\prime },j\right) $\footnote{%
Se dice entonces que $\left( j,j^{\prime }\right) $ y $\left( j^{\prime
},j\right) $ son representaciones con {\em paridad opuesta}.}. El mismo
resultado es v'alido para la acci'on de ${\bf I}_{t}$.

Esto muestra que el espacio de representaci'on correspondiente a las {\em %
representaciones irreducibles} del grupo de Lorentz completo\footnote{%
Es decir, incluyendo reflexiones espaciales y temporales.} es

\begin{itemize}
\item  la {\em suma directa} de los espacios de representaci'on de las
representaciones $\left( j,j^{\prime }\right) $ y $\left( j^{\prime
},j\right) ,$ cuando $j\neq j^{\prime }$.

\item  el espacio de representaci'on de $\left( j,j\right) $, si $%
j=j^{\prime }$.
\end{itemize}

A partir de (\ref{pasa1}), (\ref{pasa2}) y de las expresiones (\ref{x1})--(%
\ref{y2}) para las representaciones $\left( j,j^{\prime }\right) $ es
posible probar que las expresiones de ${\bf P}_{+}$, ${\bf T}_{+}$ e ${\bf I}%
_{+}$ para representaciones irreducibles del grupo completo de Lorentz
pueden considerarse tal como se describe en los dos casos siguientes \cite
{Gilmore}

\subsubsection{Caso $j=j^{\prime }$.}

En este caso, la representaci'on tiene dimensi'on $(2j+1)^2$ y
constituye una representaci'on verdadera de $L$ ($j+j^{\prime }=2j=$
entero). La acci'on de ${\bf P}_{+}$, ${\bf T}_{+}$ e ${\bf I}_{+}$
vienen dadas por 
\begin{equation}
{\bf P}_{+}\left| 
\begin{array}{c}
jj \\ 
mm^{\prime }
\end{array}
\right\rangle =\beta _{s}^{jj}\left| 
\begin{array}{c}
jj \\ 
m^{\prime }m
\end{array}
\right\rangle , 
\end{equation}
\begin{equation}
{\bf T}_{+}\left| 
\begin{array}{c}
jj \\ 
mm^{\prime }
\end{array}
\right\rangle =\beta _{t}^{jj}\left| 
\begin{array}{c}
jj \\ 
m^{\prime }m
\end{array}
\right\rangle , 
\end{equation}
\begin{equation}
{\bf I}_{+}\left| 
\begin{array}{c}
jj \\ 
mm^{\prime }
\end{array}
\right\rangle =\beta _{s}^{jj}\beta _{t}^{jj}\left| 
\begin{array}{c}
jj \\ 
mm^{\prime }
\end{array}
\right\rangle , 
\end{equation}
donde los n'umeros $\beta _{s}^{jj}$ y $\beta _{t}^{jj}$ pueden asumir
independientemente los valores $1$ 'o $-1$. En este caso se encuentran
cuatro posibles representaciones inequivalente de la extensi'on $%
\widetilde{L}_{8}$. Las matrices que representan a los generadores de
rotaciones y boosts tienen la misma expresi'on que en la
representaci'on $\left( j,j\right) $ de $SL(2,C)$.

\subsubsection{Caso $j\neq j^{\prime }$.}

Aqu{\'\i} las representaciones irreducibles tienen dimensi'on $2\left(
2j+1\right) \left( 2j^{\prime }+1\right) $ y s'olo constituyen
representaciones verdaderas de $L$ si $j+j^{\prime }$ es entero. En caso
contrario se obtiene una representaci'on proyectiva.

\paragraph{${\bf j+j}^{\prime }{\bf \neq }$entero.}

Aqu{\'\i} ${\bf P}_{+}$, ${\bf T}_{+}$ e ${\bf I}_{+}$ se pueden considerar
de la forma\footnote{${\bf 1}$ denota la identidad de dimensi'on $\left(
2j+1\right) \left( 2j^{\prime }+1\right) $.} 
\begin{equation}
{\bf P}_{+}=i\left( 
\begin{array}{cc}
{\bf 0} & {\bf 1} \\ 
{\bf 1} & {\bf 0}
\end{array}
\right) ,\qquad {\bf T}_{+}=\left( 
\begin{array}{cc}
{\bf 0} & {\bf 1} \\ 
-{\bf 1} & {\bf 0}
\end{array}
\right) ,\qquad {\bf I}_{+}=-i\left( 
\begin{array}{cc}
{\bf 1} & {\bf 0} \\ 
{\bf 0} & -{\bf 1}
\end{array}
\right) . 
\end{equation}

\paragraph{${\bf j+j}^{\prime }{\bf =}$entero.}

Aqu{\'\i} ${\bf P}_{+}$, ${\bf T}_{+}$ e ${\bf I}_{+}$ se pueden considerar
de dos maneras inequivalentes 
\begin{equation}
{\bf P}_{+}=\left( 
\begin{array}{cc}
{\bf 0} & {\bf 1} \\ 
{\bf 1} & {\bf 0}
\end{array}
\right) ,\qquad {\bf T}_{+}=\left( 
\begin{array}{cc}
{\bf 0} & {\bf 1} \\ 
{\bf 1} & {\bf 0}
\end{array}
\right) ,\qquad {\bf I}_{+}=\left( 
\begin{array}{cc}
{\bf 1} & {\bf 0} \\ 
{\bf 0} & {\bf 1}
\end{array}
\right) , 
\end{equation}
o bien 
\begin{equation}
{\bf P}_{+}=\left( 
\begin{array}{cc}
{\bf 0} & {\bf 1} \\ 
{\bf 1} & {\bf 0}
\end{array}
\right) ,\qquad {\bf T}_{+}=\left( 
\begin{array}{cc}
{\bf 0} & -{\bf 1} \\ 
-{\bf 1} & {\bf 0}
\end{array}
\right) ,\qquad {\bf I}_{+}=\left( 
\begin{array}{cc}
-{\bf 1} & {\bf 0} \\ 
{\bf 0} & -{\bf 1}
\end{array}
\right) . 
\end{equation}

En cualquiera de estos casos, los operadores de rotaciones y boosts,
denotados conjuntamente por ${\bf T}_{\alpha }$, vienen dados por la suma
directa de las matrices correspondientes a las reps. $\left( j,j^{\prime
}\right) $ y $\left( j^{\prime },j\right) $, es decir 
\begin{equation}
{\bf T}_{\alpha }=\left( 
\begin{array}{cc}
{\bf T}_{\alpha }^{\left( j,j^{\prime }\right) } & {\bf 0} \\ 
{\bf 0} & {\bf T}_{\alpha }^{\left( j^{\prime },j\right) }
\end{array}
\right) . 
\end{equation}

As{\'\i}, cuando se incluyen las reflexiones espaciales y temporales, para
considerar representaciones del grupo completo de Lorentz $L$, se encuentra
que cada una de las 8 extensiones inequivalentes admite, en general, m'as
de una representaci'on.

\subsection{Ejemplos}

\subsubsection{$j=j^{\prime }={1}/{2}.$}

La extensi'on $\widetilde{L}_{8}$ de $SL(2,C)$, nos entrega, para el caso 
$j=j^{\prime }={1}/{2}$, las siguientes posibles representaciones 
\begin{equation}
{\bf P}_{+}=\pm \left( 
\begin{array}{cccc}
1 & 0 & 0 & 0 \\ 
0 & 0 & 1 & 0 \\ 
0 & 1 & 0 & 0 \\ 
0 & 0 & 0 & 1
\end{array}
\right) ,\qquad {\bf I}_{+}=\pm \left( 
\begin{array}{cccc}
1 & 0 & 0 & 0 \\ 
0 & 0 & 1 & 0 \\ 
0 & 1 & 0 & 0 \\ 
0 & 0 & 0 & 1
\end{array}
\right) ,\qquad {\bf I}_{+}={\bf T}_{+}{\bf I}_{+}. 
\end{equation}

La representaci'on original se obtiene en el caso 
\begin{equation}
{\bf P}_{+}=\left( 
\begin{array}{cccc}
-1 & 0 & 0 & 0 \\ 
0 & 0 & -1 & 0 \\ 
0 & -1 & 0 & 0 \\ 
0 & 0 & 0 & -1
\end{array}
\right) ,\qquad {\bf I}_{+}=\left( 
\begin{array}{cccc}
1 & 0 & 0 & 0 \\ 
0 & 0 & 1 & 0 \\ 
0 & 1 & 0 & 0 \\ 
0 & 0 & 0 & 1
\end{array}
\right) ,\qquad {\bf I}_{+}={\bf T}_{+}{\bf I}_{+}, 
\end{equation}
y las usuales matrices de la sec. \ref{descom} se encuentran aplicando la
transformaci'on de similaridad de la sec. \ref{vect}.

\subsubsection{$j=0$, $j^{\prime }=\frac{1}{2}.$}

En este caso, la extensi'on $\widetilde{L}_{8}$ define las matrices
correspondientes a las transformaciones discretas de Lorentz 
\begin{equation}
{\bf P}_{+}=i\left( 
\begin{array}{cc}
{\bf 0} & {\bf 1}_{2} \\ 
{\bf 1}_{2} & {\bf 0}
\end{array}
\right) ,\qquad {\bf T}_{+}=\left(
\begin{array}{cc}
{\bf 0} & {\bf 1}_{2} \\ 
-{\bf 1}_{2} & {\bf 0}
\end{array}
\right) ,\qquad {\bf I}_{+}=-i\left( 
\begin{array}{cc}
{\bf 1}_{2} & {\bf 0} \\ 
{\bf 0} & -{\bf 1}_{2}
\end{array}
\right) . 
\end{equation}

